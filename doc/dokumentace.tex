\documentclass[a4paper]{article}
\usepackage[utf8]{inputenc}
\title{IFJ 2014 - Dokumentace}
\author{Jásenský Michal, Kulda Jiří, Lukeš Petr, Nechutný Stanislav}
\date{\today}
\begin{document}

% Uvodni stranka
\begin{center}
	\begin{LARGE}IFJ 2014/2015\end{LARGE}

	\begin{Large}Tým XX, varianta YY\end{Large}
	\\ [6in]
	
	\today
\end{center}

	xjasen00 - Jásenský Michal (25\%)

	xkulda00 - Kulda Jiří  (25\%)
	
	xlukes00 - Lukeš Petr  (25\%)
	
	xnechu01 - Nechutný Stanislav  (25\%)


	


\pagebreak

%%%%%%%%%%%%%%%%%%%%%%%%%%%%%%%%%%%%%%%%%%%%%%%%%%%%%%%%%%%%%%%%%%%%%%%%%%%%%%
\section{Lexikální analyzátor} \label{lexika}
%%%%%%%%%%%%%%%%%%%%%%%%%%%%%%%%%%%%%%%%%%%%%%%%%%%%%%%%%%%%%%%%%%%%%%%%%%%%%%

Lexikální analyzátor je implementován konečným automatem, který načítá zdrojový kód ze souboru a 
čte z něj jednotlivé tokeny. 

Vstupní soubor získá lexikální analyzátor z globální proměnné \textit{global} popsané v \textit{garbage.h}. 
Je volán syntaktickým analyzátorem, kterému vrací strukturu \textit{TToken} obsahující typ tokenu a v případě 
textového řetězce, klíčového slova a číslice obsahuje jěště hodnotu.

Syntaktický analyzátor volá funkci \textit{token\_get}. Ta pomocí funkce \textit{token\_init} naalokuje 
token. Dále mu přiřadí odpovídající hodnotu  podle načteného lexému ze zdrojového souboru a vrátí jej. 
Syntaktický analyzátor dále používá další dvě funkce. \textit{token\_return_token} slouží k navrácení již 
načteného tokenu k opětovnému načtení při dalším volání funkce \textit{token\_get}. \textit{token\_free} uvolní strukturu tokenu z paměti.

Zjednodušený obrázek konečného automatu naleznete na poslední straně(\ref{konecny automat}).

\end{document}
